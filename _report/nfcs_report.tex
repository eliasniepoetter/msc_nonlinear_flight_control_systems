\documentclass[10pt,a4paper]{article}
\usepackage[left=2.0cm,right=2.0cm,top=3.5cm,bottom=1.5cm]{geometry}
\usepackage[utf8]{inputenc}
\usepackage[english]{babel}
\usepackage[T1]{fontenc}
\usepackage{amsmath}
\usepackage{amsfonts}
\usepackage{amssymb}
\usepackage{graphicx}
\usepackage{lmodern}
\usepackage{siunitx}
\usepackage{fancyhdr}
\usepackage{enumerate}
\usepackage{mathtools}
\usepackage{float}
\usepackage{csquotes}
\usepackage{multicol}
\usepackage{lipsum}
\usepackage{tcolorbox}
\usepackage{enumitem}
\usepackage{varwidth}

% Setup SI units
\sisetup{locale=DE}
\sisetup{per-mode = symbol-or-fraction}
\sisetup{separate-uncertainty=true}
\DeclareSIUnit\year{a}
\DeclareSIUnit\clight{c}
 
 % Einbinden von Bildern
\usepackage{float}
\usepackage{graphicx, subfigure}	
	\graphicspath{{img/}}
\usepackage[skip=2pt]{caption}
 
% Abkürzungen
\usepackage{glossaries}
\newacronym{mpc}{MPC}{Model Predictive Control}
\newacronym{sos}{SOS}{Sum of Squares}



 
% Being of document
%----------------------------------------------------------------------

% Defintions for the Document
\begin{document}
\pagestyle{fancy}
\lhead{University of Stuttgart}
\rhead{Institute of Flight Mechanics and Control}

% space around floating objects
\setlength\intextsep{5pt}
\setlength{\belowcaptionskip}{0pt}

% parskip
\setlength{\parindent}{0pt}

%----------------------------------------------------------------------

\begin{center}
	\LARGE{\textbf{Systems Theory Project Report}}\\[0.25em]
	\normalsize\textbf{Nonlinear Model Predictive Control Design}\\[0.25em]
	\normalsize{Elias Niepötter (3684096)}
\end{center}

\vskip 0.5cm

\begin{abstract}
	\lipsum[1]
\end{abstract}

\vskip 1cm

\begin{multicols}{2}

\section{Introduction}
The goal of the project is to combine the methods of \gls{sos} Programming and the design of nonlinear \gls{mpc}. Based on an exemplary nonlinear dynamical system
an \gls{mpc} is designed in such a way that it acts outside a specified region in the state space and drives it towards this region.
The region is called \textit{terminal region}. Inside the terminal region a lienar controller takes over, this concept is called \textit{Dual Model Control}.
\gls{sos} methods are used to estimate the terminal region which is a \textit{region of attraction} of the closed loop dynamics of the linearly controlled 
system.

The report first introduces the basics of \gls{mpc} in \ref{sec:mpc}. Continuing with the concept of dual mode controller design focussing on \gls{sos} methods in \ref{sec:dualModeControl}.
Lastly, the methods presented are applied to the ball beam system. The results are shown and discussed in \ref{sec:example}.

\section{Model Predictive Control}
\label{sec:mpc}
\lipsum[1-2]

\section{Dual Mode Control}
\label{sec:dualModeControl}
\lipsum[1-2]

\section{Exemplary Application}
\label{sec:example}
\lipsum[1-2]


\end{multicols}
\end{document}